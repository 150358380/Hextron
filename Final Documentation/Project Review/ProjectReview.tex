\documentclass[12pt, titlepage]{article}
\usepackage{indentfirst}
\usepackage{booktabs}
\usepackage{tabularx}
\usepackage{hyperref}
\hypersetup{
    colorlinks,
    citecolor=black,
    filecolor=black,
    linkcolor=black,
    urlcolor=blue
}
\usepackage[round]{natbib}

\title{SE 3XA3: Project Review}

\author{Team \#3 - Hextron
		\\ Jason Li lij107
		\\ Yousaf Shaheen shaheeny
		\\ Scott Williams willis12
}

\date{December 6, 2017}


\begin{document}

\maketitle

\pagenumbering{roman}


\newpage 

\section{Project Review}

\indent What the Hextron group learned from this project is how everything in the software development cycle is synthesized to create a fully functional product. The development team really worked together well in order to ensure that all elements of the project came together towards the end. However, there were some roadblocks along the way that, while they created a significant difficulty hurdle, made sure that the project was completed to a high standard.\\
\newline 
\indent The implementation aspects were really difficult at the start of the project because two team members were unfamiliar with Unity to any degree. Only one team member was familiar with it, so we had to ensure that we had some knowledge of Unity before going deep into the project. Later in the project, floodfill algorithm implementations for destroying trapezoids as well as camera functionality were major complications in getting the Hextron game to work. Once those problems were alleviated, the implementation proved to be straightforward.\\

\indent It is also clear that the documentation was fairly difficult. The reason for this, as made clear in the final documentation, is that every document needs to be traced between each other. There must be a clear level of traceability between the standard requirements document, in addition to the test plan, Module Guide, and the final test report. For this, the entire group did our utmost to check whether or not requirements were fulfilled or not. After scanning the requirements document, test plan, and module hierarchies  inside the Module Guide, we could say that we accomplished what was necessary.\\
\newline 
\indent Overall, Hextron broadened our knowledge of the software development cycle. It put our group into a scenario that mocked real-world scenarios in the workplace from requirements to testing. This project felt like it was valuable, and provided a sense of motivation for us to do what was necessary. We wanted to make sure that every aspect of our project was complete from a writing and visual standpoint. It also helped that our group really collaborated well together and that all of us contributed to a great degree on different parts of the project! This helped maximize our efficiency, and made sure that the Hextron product was efficient itself! The development team loved working on Hextron, and hope that every stakeholder loves it too! 
\end{document} 